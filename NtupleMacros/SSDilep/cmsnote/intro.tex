\section{Introduction}
\label{sec:intro}
In this note we describe data driven methods to estimate the backgrounds in 
searches for new physics in events with two high $P_T$, isolated, same sign leptons,
large \met, and  significant hadronic activity.  This generic signature is sensitive to several
new physics scenarios such as SUSY.  For the purpose of this note we restrict
ourselves to the $ee$, $e\mu$, and $\mu\mu$ final states, {\em i.e.}, we 
do not consider $\tau$'s, except in the case that the $\tau$ decays leptonically.

For a reasonable event selection, as shown in Section~\ref{sec:yields}, the dominant
background is from \ttbar events. We categorize the total background into contributions
from charge mis-measurement of leptons and events with fake leptons. The former is estimated 
using charge fake probability, as discussed in Section~\ref{sec:chargemisid}. The latter, which
also includes  heavy flavor decays is estimated using lepton fake rate 
method described in Section~\ref{sec:leptonfake}.

This note is organized as follows. In Section~\ref{sec:datasamples} we list the 
Monte Carlo data samples, as well as the software tags used 
in the analysis; in Section~\ref{sec:eventselection} we describe the same sign dilepton event 
selection used in this study;  the expected event yields for the dominant 
Standard Model processes, as well as a few SUSY benchmark points, are given in Section~\ref{sec:yields}; 
a brief introduction to classifying the dominant background based on their origin
is given in Section~\ref{sec:bkgtypes}; in Section~\ref{sec:chargemisid} we discuss the data driven 
procedure to estimate the charge mis-measurement followed by lepton fake rate method in 
Section~\ref{sec:leptonfake}; we apply the above mentioned data driven methods to the Standard Model 
cocktail in Section~\ref{sec:application} and study the effect of applying the background
prediction procedure in the presence of new physics; finally, in Section~\ref{sec:conclusion}
we summarize the results.  

