\documentclass{beamer}
\usepackage{graphicx}
\usepackage{tikz}
%\usepackage[pdftex]{graphicx}
%\graphicspath{{./plots/}{./}} %{./figuresdir3/}}

%\pgfpagesuselayout{resize to}[letterpaper,landscape] 
%\usepackage[latin1]{inputenc} %only need if not in english
\usetheme{Warsaw}
\setbeamertemplate{navigation symbols}{} %no navigation

%this is for putting something on the side
%\setbeamersize{sidebar width left=0cm}
%\setbeamersize{sidebar width right=0cm}

%this is for changing the margin
\setbeamersize{text margin left=0cm}
\setbeamersize{text margin right=0cm}

%this put a page number, but nuked everything else
%\setbeamertemplate{footline}{\insertpagenumber }
%these two lines seem to magically make the footer look good and include page numbers
\useoutertheme{infolines}
\usetheme[height=7mm]{Rochester}

% talk content


\title[Isolation for Low $dR$ Dileptons]{{\huge Isolation for Low $dR$ Dileptons}\\Track and Ecal Isolation for High $P_T$ Zs}
%\title[High $P_T$ Z Efficiency]{High $P_T$ Z Efficiency\\Electron Track and Ecal Isolation}
%\title{High Pt Z Efficiency}

\author[Warren Andrews]{{\bf W. Andrews,} D. Evans, F. Golf, J. M\"ulmenst\"adt, S. Padhi, Y. Tu, F. W\"urthwein, A. Yagil -- UCSD\\
D. Barge, C. Campagnari, P. Kalavase, D. Kovalskyi, V. Krutelyov, J. Ribnik -- UCSB\\
L. Bauerdick, I. Bloch, K. Burkett, I. Fisk, O. Gutsche -- FNAL}
\institute[UCSD]{}
\date{August 31, 2009}
\begin{document}

\begin{frame}
\titlepage
\end{frame}


\begin{frame}{Outline}
  \begin{itemize}
  \item Motivation: avoid dilepton isolation self-veto in case of low $dR$ between dileptons (fratricide)
    \begin{itemize}
    \item Use case of \textcolor{red}{boosted Zs} because samples are easily available and physics is simple
    \item Consider electrons and muons seperately
    \end{itemize}
  \item Background: kinematic plots to show scale of problem
  \item Electron and muon isolation efficiencies vs dR between status 1 leptons
  \item Separate isolation components' efficiencies
    \begin{itemize}
    \item Ecal
    \item Tracker
    \item Hcal
    \end{itemize}
  \item \textbf{Track isolation fix} for electrons and muons
  \item \textbf{Ecal fix using jurassic isolation for electrons}
  \item Correct all cases: ee, $\mu\mu$, e$\mu$, multi-lepton
  \end{itemize}
\end{frame}


\begin{frame}{Brief Statement of Problem}
  \begin{columns}
    %\begin{column}{5.5cm}
    \begin{column}{0.5\linewidth}
      \begin{itemize}
      \item Goal is to avoid hypothesis dilepton self-veto (fratricide)
      \item For track isolation, if tracks are within each other's isolation cone, they will nearly always self veto
        \begin{itemize}
        \item $dR < 0.3$ for track
        \item Same for els, mus
        \end{itemize}
      \item For ecal isolation, if the jurassic region overlaps, there is potential for self veto (will return to later)
        \begin{itemize}
        \item $dR < 0.4$ {\small{for ecal outer cone}}\\ so overlap begins at $dR = 0.8$
        \item Electrons only
        \end{itemize}
      \end{itemize}
      
    \end{column}
    
    %\begin{column}{7cm}
    \begin{column}{0.5\linewidth}
      \includegraphics[width=0.95\linewidth]{talk3/outside_cone.png}
%change this picture%%%%%%%%%%%%%%%%%%%%%%%%%%%%%%%%%%%%%%%%%%%%%%%%%%%%%%%%%%

      \includegraphics[width=0.95\linewidth]{talk3/cone_border.png}
    \end{column}
  \end{columns}
\end{frame}


\begin{frame}{Kinematic Plots}
  %\tikz \draw [overlay,help lines,blue] (0,0) grid (3,2);
  %\tikz \draw [help lines] (3,2) grid (6,4);
  %\tikz{\draw[green] (15,2) -- (15,3);}
  %\tikz{\draw[cyan,overlay] (0,0) -- (5,5);}
  %\tikz{\draw[red,overlay] (canvas cs:x=6cm,y=-2cm) -- (canvas cs:x=4cm,y=-3cm);}
  %\tikz{\draw[red] (6cm,2cm) -- (6cm,3cm);}
  %\tikz{\draw[red] (10cm,2cm) -- (10cm,5cm);}
  \begin{columns}
    \begin{column}{6cm}%{5.5cm} 
      \vspace{-1cm}
      \begin{itemize}
        %\item Samples: $/ZeeJet\_Pt80to120/Summer08\_IDEAL\_V9\_v1/GEN-SIM-RECO$ ... $/ZeeJet\_Pt300toInf/Summer08\_IDEAL\_V9\_v1/GEN-SIM-RECO$
      \item Samples: \\ {\tiny{/ZeeJet\_Pt80to120/Summer08\_IDEAL\_V9\_v1/} \\ \normalsize{$\cdots$} \\ \vspace{-0.1cm}\tiny{/ZeeJet\_Pt300toInf/Summer08\_IDEAL\_V9\_v1/}}
        \\ {\tiny{/ZmumuJet\_Pt80to120/Summer08\_IDEAL\_V9\_v1/} \\ \normalsize{$\cdots$} \\ \vspace{-0.1cm}\tiny{/ZmumuJet\_Pt300toInf/Summer08\_IDEAL\_V9\_v1/}}
        \bigskip
      \item These plots are meant to set the scale of the effect
        \begin{itemize}
        %\item not large, but could be important for new physics
        \item For Z, very large boost required
        \end{itemize}
        \item Low $dR$ multilepton events could have new physics sources
        %\item Low $dR$ only occurs at high $p_T$ \\ high Z $p_T$ = high lepton $p_T$
      \end{itemize}
    \end{column}
    
    \begin{column}{6cm}
      %put a blank line between the \includegraphics to make them align vertically
      \includegraphics[width=1.0\linewidth]{plots/e_dr_final.png}
      \vspace{-0.cm}
      \includegraphics[width=1.0\linewidth]{plots/zpt_dr_final.png} %was zpt_2.png
    \end{column}
    
  \end{columns}
  %drawing it down here puts it on top of the pictures
  \tikz{\draw[->,red,overlay] (canvas cs:x=7.48cm,y=0.5cm) -- (canvas cs:x=7.48cm,y=3.55cm);}
\end{frame}


%%%!!! one MUST use 'containsverbatim' with \verb
\begin{frame}[containsverbatim]{Isolation Efficiency}
  \begin{itemize}

  \item Isolation definition\\ ISO $\equiv$ $p_T/(p_T + iso\_var)$\\ \footnotesize{ $iso\_var =$ hcal iso ~OR~ ecal iso ~OR~ track iso} 
    \\ For ecal iso only, use 2 GeV pedestal subtraction because iso calculated from hits
    \\ Ex: ISO = \verb=pt/(pt + max( els_ecal_Iso - 2.0, 0.0 ) );=
  \item \textcolor{red}{Denominator:}\\ Two status = 1 leptons \\ $p_T > 20$ GeV\\ $|\eta| < 2.4$\\ $76 < m_{ll} < 106$ GeV
    %\item Numerator:\\ $dR$ match (0.015) status 1 to reco\\ $p_T/(p_T + iso\_var) > 0.92$\\ \footnotesize{ $iso\_var =$ els\_pat\_hcalIso() + els\_pat\_ecalIso() + els\_pat\_trackIso()}
    %\item \textcolor{red}{put dr match in denominator???}
  \item \textcolor{red}{Numerator:}\\ $dR$ match (0.015) status 1 to reco\\ ISO $> iso\_cut$
  \item Plot isolation efficiency (= Numerator divided by Denominator) versus $dR$
    %\begin{itemize}
    %\scriptsize{
    %}
    %\end{itemize}
  \item This is used to define a control region: $0.5 < dR < 1.0$ which is used for individual isolations to determine the value of $iso\_cut$.
  \end{itemize}
\end{frame}


\begin{frame}{Isolation Components}
  %\begin{columns}

  %\begin{column}{0.5\textwidth}
  \begin{itemize}
  \item Study the three isolation components (ecal, hcal, track) independently
  \item Electrons
    \begin{itemize}
      %\item CandIsolatorFromDeposits
    \item \scriptsize{EleIsoTrackExtractorBlock}
    \item \scriptsize{EleIsoEcalFromHitsExtractorBlock}
    \item \scriptsize{EleIsoHcalFromTowersExtractorBlock}
    \end{itemize}
  \item Muons
    \begin{itemize}
    \item \scriptsize{\texttt{edm::View<Muon> muon->isolationR03().sumPt}}
    \item \texttt{edm::View<Muon> muon->isolationR03().emEt}
    \item \texttt{edm::View<Muon> muon->isolationR03().hadEt}
    \end{itemize}
  \item Set common isolation efficiency operating point: 
  \item that is, find a point for each such that each efficiency is roughly the same
    \begin{itemize}
    %\item Plot isolation components separately:
      %\begin{itemize}
        %\item $els\_pat\_hcalIso()$
        %\item $els\_pat\_ecalIso()$
        %\item $els\_pat\_trackIso()$
      %\end{itemize}
    \item Plot only for status 1 leptons within $0.5 < dR < 1.0$
    \item Find 90\% efficiency point
    \item This defines the cut for that isolation variable used in numerator of individual isolation efficiencies
    \end{itemize}
  \end{itemize}
  %\end{column}
  
  %\begin{column}{0.55\textwidth}
  %  \includegraphics[width=0.95\linewidth]{plots/ZeeJetALL80toInf_nofilter_e_drstat1_iso_eff.png}
  %
  %  \includegraphics[width=0.95\linewidth]{plots/ZeeJetALL80toInf_nofilter_e_drstat1_iso_eff.png}
  %\end{column}
  
  %\end{columns}
\end{frame}

%make this e1 and e2, axis label, range up to 1.1 or 1.2
\begin{frame}{Isolation Component Efficiencies: Electrons}
  \begin{center}
    %The three component's distributions are different at low $dR$.\\ \textcolor{red}{fix legend, axis labels}\bigskip
    
    %\includegraphics[width=0.95\linewidth]{plots/ZeeJetALL80toInf_nofilter_e1_drstat1_tri_iso_comp.png}
\includegraphics[width=0.95\linewidth]{plots/ele_tri_iso_comp_final.png}

    Track and ecal isolation need fix for low $dR$, hcal is ok as is.
  \end{center}
\end{frame}

%same here as above
\begin{frame}{Isolation Component Efficiencies: Muons}%{Muon Isolation Efficiencies}
  %Pick cut by finding 95\% efficiency point, get $\sim0.98$ for all three \\ \medskip
  %\begin{columns}%[T]
    
  %\begin{column}{0.53\textwidth}
  \begin{center}
  %Leading $p_T$ muon
    %\includegraphics[width=0.95\linewidth]{plots/ZmmJetALL80toInf_nofilter_m1_drstat1_tri_iso_comp.png}
    \vspace{-0.2cm}
    \includegraphics[width=0.95\linewidth]{plots/mus_tri_iso_comp_final.png}
    
  %\vspace{-0.3cm}{\footnotesize{$dR$ between status 1 muons}}
  %\end{center}
  %\end{column}
    
  %\begin{column}{0.53\textwidth}
  %\begin{center}
  %{\footnotesize{This is the efficiency to isolate and $dR$ match both electrons.}} \\ \medskip
  %{\footnotesize{\textcolor{red}{NEW}}} \\
  %  \includegraphics[width=1.0\linewidth]{plots/ZeeJetALL80toInf_nofilter_e_drstat1_ecal_iso_pair_eff_comp.png}
  %Sub-leading $p_T$ muon
  %\includegraphics[width=1.0\linewidth]{plots/ZmmJetALL80toInf_nofilter_m2_drstat1_tri_iso_comp.png}

  %\vspace{-0.3cm}{\footnotesize{$dR$ between status 1 muons}}
  %\end{center}
  %\end{column}
    
  %\end{columns}
  %\begin{center}
  Loss of efficiency at low $dR$ is appreciable only for track isolation, not ecal or hcal. For muons, only need to improve the track isolation.
  \end{center}
\end{frame}


%*********change picture so cone is fully inside
\begin{frame}{An Improved Track Isolation Algorithm}%{An Affable Track Isolation Algorithm}
  \begin{itemize}
    \item This is the fix to the \textcolor{red}{track} isolation.
    \item Same for electrons and muons (except for inner veto cone size)
    \item Still apply same numerator and denominator, but with the addition:
    %\item If second electron (e2) is within cone of first (e1), then exclude all tracks within $dR$ of $0.015$ of the second electron (e2) from the tracks used to calculate isolation for the first electron (e1)
    \item If a second electron (e2) is within the cone of first (e1), then exclude all tracks within $dR$ of $0.015$ of e2 from the tracks used to calculate isolation for e1.
    %\item Track and lepton quality cuts chosen by user (standard default)
    %\item if the second electron also $dR$ matched to status 1, exclude tracks within $dR$ of $0.015$ around electron momentum at vertex from the isolation.
    %\item In other words, add to numerator: \\
      %\hspace{0.5cm}\textcolor{red}{$dR > 0.015$ from second electron} \\ \bigskip
    %\item For use without MC, just require id on second electron. Here we want just to know the baseline efficiency to start.
  \end{itemize}

  \begin{center}
    %\vspace{-0.1cm} 
    \scriptsize{e2} \hspace{1.75cm} \scriptsize{e1}\\
    \vspace{-0.025cm} %\includegraphics[width=0.4\linewidth]{talk3/jurassic_border.png}\\
    \includegraphics[width=0.4\linewidth]{talk3/cone_border.png} \\
    Tracks in e2 inner \textcolor{red}{veto cone} are excluded from isolation of e1.
  \end{center}

\end{frame}


\begin{frame}{Improved Track Isolation Results: Electrons}
  \begin{center}
    %\begin{columns}%[T]
    %\begin{column}{0.3\textwidth}
    %These are for leading and second electron separately.\\ \bigskip Note that if the second electron is not $dR$ matched, the old isolation is used. %\\ \bigskip To be fixed: instead of truth matching, exclude all ``good'' electrons
    %\end{column}

    %\begin{column}{0.6\textwidth}
    %\includegraphics[width=0.95\linewidth]{plots/ZeeJetALL80toInf_nofilter_e1_drstat1_trck_iso_recalc_eff_comp.png}
    \vspace{-0.2cm}
    %\includegraphics[width=0.95\linewidth]{plots/ZeeJetALL80toInf_nofilter_e2_drstat1_trck_iso_recalc_eff_comp.png}
    \includegraphics[width=0.95\linewidth]{plots/ele_trck_iso_comp_final.png}
    \vspace{-0.1cm}
    The algorithm recovers the efficiency loss at low $dR$.\\Note that the efficiency includes reconstruction and isolation.
    %\end{column}
    %\end{columns}
  \end{center}
\end{frame}


\begin{frame}{Improved Track Isolation Results: Muons}%{Muon Track Isolation Fix}
  \begin{center}
    %Same as for electrons, except inner cone is $dR < 0.01$
    \vspace{-0.2cm}
    \includegraphics[width=0.95\linewidth]{plots/mus_track_isolation_final.png}
    \vspace{-0.1cm}
    Efficiency loss is again recovered.
    %{\footnotesize{$dR$ between status 1 muons}}
  \end{center}
  %\end{column}
  %\end{columns}
\end{frame}

%%%%%%%%%%%%%%%%%%%%%%%%%%%%%%%%%%%
% shade exclusion region
\begin{frame}{An Improved \textcolor{red}{Ecal} Isolation Algorithm}
  \vspace{-0.2cm}
  \begin{itemize}
    \item For electrons only
    \item The general idea is the same as the track algorithm: exclude second electron from isolation cone of first
    \item Details: %{\small{Details:}} 
      \begin{itemize}
      \item Keep the same jurassic exclusion (LCone): outer radius = 0.4, inner exclusion = 0.07, jurassic strip is full cone width and $\eta = 0.04$ wide. Using hits.
      \item New exclusion is the same jurassic region around other electrons. Even if only a small piece of the second region is overlapping, still exclude hits in it.
      \end{itemize}
      %\item Do this for two cases: ``other electrons'' means everything in the gsf electron block, and only for gsf electrons with $p_t > 15$ and $H/E > 0.1$.
      %\item {\small{We also tried excluding only the supercluster energy instead of the jurassic region of any electrons which overlap the outer cone.}}
      %\item I'll just show the jurassic exclusion with the cuts, but others are in backup slides.
      %\item This is done only for the \textcolor{red}{ZeeJet\_Pt300toInf} sample% because I have to re-ntuple to change the code. But this is the only sample in which $dR$ gets down to 0.5 anyway.%{\footnotesize{}}
  \end{itemize}
  \begin{center}  
    \vspace{-0.2cm} \scriptsize{e2} \hspace{1.75cm} \scriptsize{e1}\\
    \vspace{-0.025cm} \includegraphics[width=0.4\linewidth]{talk3/jurassic_border.png}\\
    \vspace{-0.2cm} e2 jurassic region excluded from isolation of e1.
  \end{center}
\end{frame}


\begin{frame}{Improved Ecal Isolation Results}
  \begin{center}
    %\begin{columns}%[T]
    %\begin{column}{0.4\textwidth}

    %\end{column} %\pause
    %\begin{column}{0.6\textwidth}
    %\includegraphics[width=0.95\linewidth]{plots/ZeeJet300toInf_nofilter_sngl_e1_drstat1_ecal_affjrcutiso_eff_comp.png}
    \vspace{-0.25cm}
    %need this plot with the pedestal subtraction, and combine the two
    %\includegraphics[width=0.95\linewidth]{plots/ZeeJet300toInf_nofilter_sngl_e2_drstat1_ecal_affjrcutiso_eff_comp.png}
    \includegraphics[width=0.95\linewidth]{plots/ecal_isolation_talk.png}
    %\end{column}
    %\end{columns}

    \footnotesize For ecal also, the improved algorithm recovers much of the efficiency loss at low $dR$. \\
    Recall, isolation = pt/(pt + ecal\_iso), ecal\_iso includes 2GeV pedestal subtration
  \end{center}
\end{frame}


\begin{frame}{Conclusion}
  \begin{itemize}
  \item We have studied the performance of existing isolation variables in the case of leptons with small $dR$ between them
    \begin{itemize}
    \item Used case of boosted Zs
    \end{itemize}
  \item In order to increase the efficiency at low $dR$, we implemented an improved algorithm which allows a second lepton in the isolation cone
    \begin{itemize}
      %\item Change track algorithm slightly so that it has requirements more similar to ecal algorithm
    \item For track isolation, electrons and muons are treated identically (except for inner cone size)
    \item For ecal isolation, only electrons need fixing, use jurassic algorithm
    \item Correct all cases: ee, $\mu\mu$, e$\mu$, multi-lepton
    \end{itemize}
  \item This recovers efficiency loss at low $dR$ for both tracker and ecal isolations
  %\item In the process, we have found another problem which is perhaps more worrisome, which is the efficiency at moderate to large $dR$. We will look into this now.
  \end{itemize}
\end{frame}


%backup slides

\begin{frame}{}
  \huge{ Backup Slides }
\end{frame}


\begin{frame}{Kinematic Plots Continued}
  \begin{columns}

    \begin{column}{0.6\textwidth}
      \includegraphics[width=0.95\linewidth]{plots/ZeeJetALL80toInf_nofilter_e_drstat1_e1_pt.png}
      \vspace{-0.2cm}
      \includegraphics[width=0.95\linewidth]{plots/ZeeJetALL80toInf_nofilter_e_drstat1_e2_pt.png}
    \end{column}
  \end{columns}
\end{frame}

\end{document}
