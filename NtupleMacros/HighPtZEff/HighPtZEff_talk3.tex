\documentclass{beamer}
\usepackage{graphicx}
%\usepackage[pdftex]{graphicx}
%\graphicspath{{./plots/}{./}} %{./figuresdir3/}}

%\pgfpagesuselayout{resize to}[letterpaper,landscape] 
%\usepackage[latin1]{inputenc} %only need if not in english
\usetheme{Warsaw}
\setbeamertemplate{navigation symbols}{} %no navigation

%this is for putting something on the side
%\setbeamersize{sidebar width left=0cm}
%\setbeamersize{sidebar width right=0cm}

%this is for changing the margin
\setbeamersize{text margin left=0cm}
\setbeamersize{text margin right=0cm}

%this put a page number, but nuked everything else
%\setbeamertemplate{footline}{\insertpagenumber }
%these two lines seem to magically make the footer look good and include page numbers
\useoutertheme{infolines}
\usetheme[height=7mm]{Rochester}

% talk content

\title[High $P_T$ Z Efficiency]{High $P_T$ Z Efficiency\\Electron Track and Ecal Isolation}
%\title{High Pt Z Efficiency}

\author{Warren Andrews, FKW, Dave}
\institute{UCSD}
\date{July 16, 2009}
\begin{document}

\begin{frame}
\titlepage
\end{frame}


\begin{frame}{Outline}
  \begin{itemize}
    \item Motivation: avoid fratricide (dilepton self-veto), use case of boosted Zs because it's simple
    \item Consider only electrons
    \item Background: kinematic plots to show scale of problem
    \item Overall isolation efficiency vs dR between status 1 leptons
    \item Separate isolation components' efficiencies
    \item \textbf{Track isolation looper level fix}
    \item \textbf{Ecal RECO level fix using jurassic isolation}
    \item Future plans
  \end{itemize}
\end{frame}


\begin{frame}{Brief Statement of Problem}
  \begin{columns}
    \begin{column}{5.5cm}
      
      \begin{itemize}
      \item Goal is to avoid hypothesis dilepton self-veto (fratricide)
      \item For track isolation, if tracks are within each other's isolation cone, they will nearly always self veto
        \begin{itemize}
        \item $dR < 0.3$ for track
        \end{itemize}
      \item For ecal isolation, if the jurassic region overlaps, there is potential for self veto (will return to later)
        \begin{itemize}
        \item $dR < 0.4$ {\small{for ecal outer cone}}\\ so overlap begins at $dR = 0.8$
        \end{itemize}
      \end{itemize}
      
    \end{column}
    
    \begin{column}{7cm}
      \includegraphics[width=0.95\linewidth]{talk3/outside_cone.png}

      \includegraphics[width=0.95\linewidth]{talk3/cone_border.png}
    \end{column}
  \end{columns}
\end{frame}


\begin{frame}{Kinematic Plots}
  \begin{columns}
    \begin{column}{5cm} 
      \begin{itemize}
        %\item Samples: $/ZeeJet\_Pt80to120/Summer08\_IDEAL\_V9\_v1/GEN-SIM-RECO$ ... $/ZeeJet\_Pt300toInf/Summer08\_IDEAL\_V9\_v1/GEN-SIM-RECO$
      \item Samples: \\ {\tiny{/ZeeJet\_Pt80to120/Summer08\_IDEAL\_V9\_v1/} \\ \normalsize{$\cdots$} \\ \tiny{/ZeeJet\_Pt300toInf/Summer08\_IDEAL\_V9\_v1/}}
        %\normalsize{\item Plotted for status 1 parameters (denominator of everything)}:
        %\begin{itemize}
        %\item Each lepton $p_T > 20$, $|\eta| < 2.4$
        %\item Dilepton mass in 96 to 106
        %\end{itemize}
        \bigskip
      \item These plots are meant to set the scale of the effect
        \begin{itemize}
        \item not large, but could be important for new physics
        \end{itemize}
      \end{itemize}
    \end{column}
    
    \begin{column}{6.55cm}
      %put a blank line between the \includegraphics to make them align vertically
      %\includegraphics[width=0.95\linewidth]{plots/ZeeJetALL80toInf_nofilter_e_drstat1.png}
      \includegraphics[width=1.0\linewidth]{plots/ZeeJetALL80toInf_nofilter_e_drstat1.png}
      %\end{column}
      \vspace{-0.25cm}
      %\begin{column}{2cm}
      %\begin{figure}
      \includegraphics[width=1.0\linewidth]{plots/ZeeJetALL80toInf_nofilter_e_drstat1_zpt_2.png}
      %\end{figure}
    \end{column}
    
  \end{columns}
\end{frame}

\begin{frame}{WW-style Isolation Efficiency}
  \begin{columns}
    %\begin{column}{6cm}
    \begin{column}{0.5\textwidth}
      \small{
      \begin{itemize}
      \item Denominator:\\ Two status = 1 electrons \\ $p_T > 20$\\ $|\eta| < 2.4$\\ $96 < m_{ll} < 106$
      \item Numerator:\\ $dR$ match (0.015) status 1 to reco\\ $p_T/(p_T + iso\_var) > 0.92$\\ \footnotesize{ $iso\_var =$ els\_pat\_hcalIso() + els\_pat\_ecalIso() + els\_pat\_trackIso()}
      \item Plot isolation efficiency versus dR
        %\begin{itemize}
          %\scriptsize{
          %\item Also plotted efficiency just for $dR$ match and for id, and all versus leading/sub-leading electron pt, eta, Z pt, etc. These are not as interesting as versus dR.
          %}
        %\end{itemize}
      \item This is used to define a control region: \textcolor{red}{$0.5 < dR < 1.0$} which is used for individual isolations
      \end{itemize}
      }
    \end{column}

    %\begin{column}{6.5cm}
    \begin{column}{0.55\textwidth}
      %\centerline{
      %put a blank line between the \includegraphics to make them align vertically
      %\includegraphics[width=0.95\linewidth]{plots/ZeeJetALL80toInf_nofilter_e_drstat1_iso_eff_rebin.png}

      \includegraphics[width=0.95\linewidth]{plots/ZeeJetALL80toInf_nofilter_e_drstat1_iso_eff.png}
      %}
    \end{column}        

  \end{columns}
\end{frame}


\begin{frame}{Isolation Components}
  %\begin{columns}

  %\begin{column}{0.5\textwidth}
  \begin{itemize}
  \item Study the three isolation components independently
  \item Set common isolation efficiency operating point: 
  \item that is, find a point for each such that each efficiency is the same
    \begin{itemize}
    \item Plot isolation components separately:
      \begin{itemize}
      \item $els\_pat\_hcalIso()$
      \item $els\_pat\_ecalIso()$
      \item $els\_pat\_trackIso()$
      \end{itemize}
    \item Plot only for status 1 \textcolor{red}{$0.5 < dR < 1.0$}
    \item Find 90\% bin low edge
    \item This defines the cut for that isolation variable used in numerator of individual isolation efficiencies
    \end{itemize}
  \end{itemize}
  %\end{column}

  %\begin{column}{0.55\textwidth}
  %  \includegraphics[width=0.95\linewidth]{plots/ZeeJetALL80toInf_nofilter_e_drstat1_iso_eff.png}
  %
  %  \includegraphics[width=0.95\linewidth]{plots/ZeeJetALL80toInf_nofilter_e_drstat1_iso_eff.png}
  %\end{column}

  %\end{columns}
\end{frame}


\begin{frame}{Isolation Components (2)}
  \begin{center}
    The three component's distributions are different at low $dR$.\\ \bigskip

    \includegraphics[width=0.9\linewidth]{plots/ZeeJetALL80toInf_nofilter_e1_drstat1_tri_iso_comp.png}
  \end{center}
\end{frame}


\begin{frame}{Correcting the Track Isolation at the Looper Level}
  \begin{columns}

    \begin{column}{0.4\textwidth}
      First step is to reproduce the pat variable in ntuple using tracks in looper. \\ \bigskip
      Track requirements: \textcolor{red}{$p_t > 1$}, \textcolor{red}{$0.015 < dR < 0.4$} from electron. \\ \bigskip
      What is plotted is the \textcolor{red}{rel\_iso variable: $p_T/(p_T + track\_iso)$} \\ \bigskip
      We don't quite understand the discrepancy, but think it's small enough to ignore. \\ \bigskip
      %Basically, the pat is applying a small cut we can't find... This makes their isolation slightly better, and efficiency slightly higher.
    \end{column}

    \begin{column}{0.58\textwidth}
      \includegraphics[width=0.95\linewidth]{plots/ZeeJetALL80toInf_nofilter_e_trck_iso_comp.png}

      \includegraphics[width=0.95\linewidth]{plots/ZeeJetALL80toInf_nofilter_e_trck_iso_dr05_1_comp.png}
    \end{column}

  \end{columns}
\end{frame}


\begin{frame}{An Affable Track Isolation Algorithm}
  \begin{itemize}
    \item This is the fix to the \textcolor{red}{track} isolation.
    \item Still apply same numerator and denominator, but with the addition:
    \item If second electron is within cone of first, then exclude all tracks within $dR$ of $0.015$ of the second electron from the tracks used to calculate isolation for the first electron
    %\item if the second electron also $dR$ matched to status 1, exclude tracks within $dR$ of $0.015$ around electron momentum at vertex from the isolation.
    %\item In other words, add to numerator: \\
      %\hspace{0.5cm}\textcolor{red}{$dR > 0.015$ from second electron} \\ \bigskip
    %\item For use without MC, just require id on second electron. Here we want just to know the baseline efficiency to start.
  \end{itemize}

  \begin{center}
    \includegraphics[width=0.4\linewidth]{talk3/cone_border.png}
  \end{center}

\end{frame}


\begin{frame}{Affable Track Isolation Results (1)}
  \begin{columns}%[T]
    \begin{column}{0.3\textwidth}
      These are for leading and second electron separately.\\ \bigskip Note that if the second electron is not $dR$ matched, the old isolation is used. \\ \bigskip To be fixed: instead of truth matching, exclude all ``good'' electrons
    \end{column}

    \begin{column}{0.6\textwidth}
      \includegraphics[width=0.95\linewidth]{plots/ZeeJetALL80toInf_nofilter_e1_drstat1_trck_iso_recalc_eff_comp.png}
      \vspace{-0.2cm}
      \includegraphics[width=0.95\linewidth]{plots/ZeeJetALL80toInf_nofilter_e2_drstat1_trck_iso_recalc_eff_comp.png}

    \end{column}
  \end{columns}
\end{frame}


\begin{frame}{Affable Track Isolation Results (2)}
  \begin{columns}%[T]
    \begin{column}{0.35\textwidth}
      These are for leading and second electron combined. Ie, the numerator is both electrons $dR$ matched and isolated.
      %\\ \bigskip
      %On the bottom is the comparison between the new algorithm and the pat. We are missing a cut on tracks, so they have higher efficiency above $dR = 0.35$.
    \end{column}

    \begin{column}{0.6\textwidth}
      \includegraphics[width=0.95\linewidth]{plots/ZeeJetALL80toInf_nofilter_e_drstat1_trck_iso_recalc_pair_eff_comp.png}
      %\vspace{-0.2cm}
      %\includegraphics[width=0.95\linewidth]{plots/ZeeJetALL80toInf_nofilter_e_drstat1_trck_iso_affable_pair_eff_comp.png}
    \end{column}
  \end{columns}
\end{frame}


\begin{frame}{An Affable \textcolor{red}{Ecal} Isolation Algorithm}
  \begin{itemize}
    \item The general idea is the same as the track algorithm: exclude second electron from isolation cone of first
    \item {\small{Details:}} 
      \begin{itemize}
        \item Keep the same jurassic exclusion (LCone): outer radius = 0.4, inner exclusion = 0.07, jurassic strip is full cone width and $\eta = 0.04$ wide. Using hits.
        \item New exclusion is the same jurassic region around other electrons. Even if only a small piece of the second region is overlapping, still exclude hits in it.
      \end{itemize}
    \item Do this for two cases: ``other electrons'' means everything in the gsf electron block, and only for gsf electrons with $p_t > 15$ and $H/E > 0.1$.
    %\item {\small{We also tried excluding only the supercluster energy instead of the jurassic region of any electrons which overlap the outer cone.}}
    %\item I'll just show the jurassic exclusion with the cuts, but others are in backup slides.
    \item This is done only for the \textcolor{red}{ZEEJet\_Pt300toInf} sample because I have to re-ntuple to change the code. But this is the only sample in which $dR$ gets down to 0.5 anyway.%{\footnotesize{}}
  \end{itemize}
\end{frame}


\begin{frame}{Affable Ecal Isolation Results (1)}
  \begin{columns}%[T]
    \begin{column}{0.4\textwidth}
      \includegraphics[width=0.95\linewidth]{talk3/jurassic_border.png}
    \end{column}
    \begin{column}{0.6\textwidth}
      \includegraphics[width=0.95\linewidth]{plots/ZeeJet300toInf_nofilter_sngl_e1_drstat1_ecal_affjrcutiso_eff_comp.png}
      \vspace{-0.2cm}
      \includegraphics[width=0.95\linewidth]{plots/ZeeJet300toInf_nofilter_sngl_e2_drstat1_ecal_affjrcutiso_eff_comp.png}
    \end{column}
  \end{columns}
\end{frame}


\begin{frame}{Ecal Isolation Bigger Picture}
  The isolation efficiency drops quite rapidly at high $dR$. Because this is a high $p_T$ sample, there are jets recoiling against the Z which the second electron can overlap with. \\ \medskip However, this is still worrisome because the efficiency is different for the ecal and tracker. \bigskip
  \begin{columns}%[T]

    \begin{column}{0.5\textwidth}
      {\footnotesize{This is the efficiency to isolate and $dR$ match the softer of the two electrons.}} \\ \medskip
      \includegraphics[width=1.0\linewidth]{plots/ZeeJetALL80toInf_nofilter_e2_drstat1_ecal_iso_eff_comp.png}
    \end{column}

    \begin{column}{0.5\textwidth}
      {\footnotesize{This is the efficiency to isolate and $dR$ match both electrons.}} \\ \medskip
      \includegraphics[width=1.0\linewidth]{plots/ZeeJetALL80toInf_nofilter_e_drstat1_ecal_iso_pair_eff_comp.png}
    \end{column}

  \end{columns}
\end{frame}


\begin{frame}{Conclusion}
  \begin{itemize}
    \item We have studied the effect of the existing isolation variable in the case of leptons with small $dR$ between them
    \item In order to improve the efficiency, we implemented an ``affable'' algorithm which allows a second lepton in the isolation cone
      \begin{itemize}
      \item Change track algorithm slightly so that it has requirements more similar to ecal algorithm
      \end{itemize}
    \item This shows promise for both the tracker and the ecal
    \item In the process, we have found another problem which is perhaps more worrisome, which is the efficiency at moderate to large $dR$. We will look into this now.
  \end{itemize}
\end{frame}


%backup slides

\begin{frame}{}
  \huge{ Backup Slides }
\end{frame}


\begin{frame}{Kinematic Plots Continued}
  \begin{columns}

    \begin{column}{0.6\textwidth}
      \includegraphics[width=0.95\linewidth]{plots/ZeeJetALL80toInf_nofilter_e_drstat1_e1_pt.png}

      \includegraphics[width=0.95\linewidth]{plots/ZeeJetALL80toInf_nofilter_e_drstat1_e2_pt.png}
    \end{column}
  \end{columns}
\end{frame}


\begin{frame}{Pair Efficiency Plots}
  \begin{columns}

    \begin{column}{0.5\textwidth}
      Numerator = two $dR$ matches (0.015), and two isolated electrons \\ \bigskip

      \includegraphics[width=0.95\linewidth]{plots/ZeeJetALL80toInf_nofilter_e_drstat1_ecal_iso_pair_eff.png}
    \end{column}

    \begin{column}{0.5\textwidth}
      \includegraphics[width=0.95\linewidth]{plots/ZeeJetALL80toInf_nofilter_e_drstat1_trck_iso_pair_eff.png}

      \includegraphics[width=0.95\linewidth]{plots/ZeeJetALL80toInf_nofilter_e_drstat1_hcal_iso_pair_eff.png}
    \end{column}

  \end{columns}
\end{frame}


\begin{frame}{Isolation Component: Track}
  \begin{columns}

    \begin{column}{0.4\textwidth}
      This is not the worst isolation (see next two slides). \\ \bigskip
      But, this is the easiest isolation to fix because it can be done at looper level using ctf tracks. For this reason, it's the first to be fixed.
    \end{column}

    \begin{column}{0.6\textwidth}
      \includegraphics[width=0.95\linewidth]{plots/ZeeJetALL80toInf_nofilter_e1_drstat1_trck_iso_eff.png}
    %\end{column}

    %\begin{column}{0.5\textwidth}
      \includegraphics[width=0.95\linewidth]{plots/ZeeJetALL80toInf_nofilter_e2_drstat1_trck_iso_eff.png}
    \end{column}
  \end{columns}

\end{frame}


\begin{frame}{Isolation Component: Ecal}
  \begin{columns}

    \begin{column}{0.5\textwidth}
      This \textcolor{red}{\textit{is}} the worst isolation. \\ \bigskip
      It is interesting at both low \textit{and} high dR, but for different reasons. \\ \bigskip
      At low $dR$, overlapping isolation is the culprit. This is expected since electrons brem, and the jurassic iso veto region extends across the entire cone of $dR = 0.4$. \\ \bigskip
      At high $dR$, we aren't totally convinced of anything because we haven't gotten to it yet, but we suspect it is because there are jets recoiling against these boosted Zs which often overlap with the softer of the two leptons. This is a todo item.
    \end{column}

    \begin{column}{0.5\textwidth}
      \includegraphics[width=0.95\linewidth]{plots/ZeeJetALL80toInf_nofilter_e1_drstat1_ecal_iso_eff.png}
    %\end{column}

    %\begin{column}{0.5\textwidth}
      \includegraphics[width=0.95\linewidth]{plots/ZeeJetALL80toInf_nofilter_e2_drstat1_ecal_iso_eff.png}
    \end{column}
  \end{columns}

\end{frame}


\begin{frame}{Isolation Component: Hcal}
  \begin{columns}

    \begin{column}{0.4\textwidth}
      We're ignoring this one because, well, what's an hcal, anyway? \\ \bigskip
      It's probably the best efficiency, or at least comparable to tracker.
    \end{column}

    \begin{column}{0.6\textwidth}
      \includegraphics[width=0.95\linewidth]{plots/ZeeJetALL80toInf_nofilter_e1_drstat1_hcal_iso_eff.png}

      \includegraphics[width=0.95\linewidth]{plots/ZeeJetALL80toInf_nofilter_e2_drstat1_hcal_iso_eff.png}
    \end{column}
  \end{columns}

\end{frame}


\begin{frame}{Affable Ecal Isolation Results (2)}
  \begin{columns}%[T]
    \begin{column}{0.3\textwidth}
      This is just excluding the supercluster energy if the supercluster is in the outer cone of 0.4.
    \end{column}

    \begin{column}{0.6\textwidth}
      \includegraphics[width=0.95\linewidth]{plots/ZeeJet300toInf_nofilter_sngl_e1_drstat1_ecal_affsciso_eff_comp.png}
      \vspace{-0.2cm}
      \includegraphics[width=0.95\linewidth]{plots/ZeeJet300toInf_nofilter_sngl_e2_drstat1_ecal_affsciso_eff_comp.png}
    \end{column}
  \end{columns}
\end{frame}


\begin{frame}{Affable Ecal Isolation Results (3)}
  \begin{columns}%[T]
    \begin{column}{0.3\textwidth}
      This is excluding the jurassic region of all electrons.
    \end{column}

    \begin{column}{0.6\textwidth}
      \includegraphics[width=0.95\linewidth]{plots/ZeeJet300toInf_nofilter_sngl_e1_drstat1_ecal_affjriso_eff_comp.png}
      \vspace{-0.2cm}
      \includegraphics[width=0.95\linewidth]{plots/ZeeJet300toInf_nofilter_sngl_e2_drstat1_ecal_affjriso_eff_comp.png}
    \end{column}
  \end{columns}
\end{frame}


\end{document}


%this is a template which apparently doesn't show up because it's after \end{document}
\begin{frame}{}
  \begin{columns}
    \begin{column}{0.55\textwidth}

    \end{column}

    \begin{column}{0.55\textwidth}

    \end{column}
  \end{columns}
\end{frame}
